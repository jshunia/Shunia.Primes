\begin{proof}
We aim to prove that for an odd integer $n > 3$ such that $2^{n-1} \equiv 1 \pmod{n}$, and $D$ being the least integer greater than 1 which does not divide $n-1$, if the congruence $2^{\lfloor\frac{n-1}{D}\rfloor} + 1 = \sum_{k=0}^{n}\binom{n}{k}2^{\lfloor\frac{k}{D}\rfloor} \pmod{n}$ holds, then $n$ is prime or GCD($a(n) - 1$, $n$) is a non-trivial factor of $n$, where $a(n) = 2^{\lfloor\frac{n-1}{D}\rfloor}$.

\textbf{Step 1: Proof that $2^{\lfloor\frac{n-1}{D}\rfloor} \not\equiv 1 \pmod{n}$:} 

Given $2^{n-1} \equiv 1 \pmod{n}$, by the properties of the order of an integer modulo $n$, the smallest $k$ such that $2^k \equiv 1 \pmod{n}$ must be $k = n-1$ or a divisor of $n-1$.

Since $\lfloor\frac{n-1}{D}\rfloor$ is strictly less than $n-1$ and $D$ does not divide $n-1$, it follows that $2^{\lfloor\frac{n-1}{D}\rfloor}$ can't be equivalent to $1 \pmod{n}$.

\textbf{Step 2: Proof by contradiction for GCD statement:}
Assume to the contrary that GCD($a(n) - 1$, $n$) is trivial, i.e., equals to 1 or $n$. 

\textbf{Case 1: GCD($a(n) - 1$, $n$) = 1:} If $a(n) - 1$ and $n$ are coprime, then there exist integers $x$ and $y$ such that $x(a(n) - 1) + yn = 1$. Since $a(n) \equiv 1 \pmod{n}$ doesn't hold (as shown before), there is a contradiction. 

\textbf{Case 2: GCD($a(n) - 1$, $n$) = $n$:} This implies that $n$ divides $a(n) - 1$, which means $a(n) \equiv 1 \pmod{n}$. However, we have already shown that $a(n) \not\equiv 1 \pmod{n}$, which leads us to a contradiction. 

Thus, in both cases, we arrive at contradictions, which implies that our initial assumption is false. Hence, GCD($a(n) - 1$, $n$) must be a non-trivial factor of $n$, if $n$ is composite.

Therefore, if the initial congruence holds, $n$ is either prime or GCD($a(n) - 1$, $n$) is a non-trivial factor of $n$.
\end{proof}