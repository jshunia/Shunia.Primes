\pdfoutput=1
\documentclass{article}
\usepackage{fullpage}
\usepackage{amsmath, amssymb, amsthm}
\usepackage[utf8]{inputenc}
\usepackage[english]{babel}
\usepackage[numbers]{natbib}
\usepackage[usenames]{color}
\usepackage{url}
\usepackage{color}
\usepackage[usenames]{color}
\usepackage[usenames]{color}
\usepackage[colorlinks=true,
linkcolor=webgreen,
filecolor=webbrown,
citecolor=webgreen]{hyperref}
\definecolor{webgreen}{rgb}{0,.5,0}
\definecolor{webbrown}{rgb}{.6,0,0}
\definecolor{red}{rgb}{1,0,0}
\usepackage{cleveref}
\usepackage{tabularx}
\usepackage{placeins}
\theoremstyle{plain}
\newtheorem{conjecture}{Conjecture}
\newtheorem{theorem}{Theorem}
\newtheorem{lemma}[theorem]{Lemma}
\newtheorem{corollary}[theorem]{Corollary}
\newtheorem{proposition}[theorem]{Proposition}
\newtheorem{remark}{Remark}
\crefname{conjecture}{Conjecture}{Conjectures}
\crefname{theorem}{Theorem}{Theorems}
\crefname{lemma}{Lemma}{Lemmas}
\crefname{proposition}{Proposition}{Propositions}
\crefname{remark}{Remark}{Remarks}
\theoremstyle{definition}
\newtheorem{definition}{Definition}
\newtheorem{notation}{Notation}
\newtheorem{example}{Example}
\newtheorem{question}{Question}
\crefname{definition}{Definition}{Definitions}
\crefname{notation}{Notation}{Notations}
\crefname{example}{Example}{Examples}
\crefname{question}{Question}{Questions}
\crefname{section}{\S}{Sections}
\newcommand{\ideal}[1]{\left\langle #1 \right\rangle}
\newcommand{\floor}[1]{\left\lfloor #1 \right\rfloor}
\newcommand{\ceil}[1]{\left\lceil #1 \right\rceil}
\newcommand{\Z}{\mathbb{Z}}
\newcommand{\K}{\mathcal{K}}
\newcommand{\wt}[1]{\#(#1)}
\newcommand{\lcm}{\text{lcm}}
\newcommand{\ord}{\text{ord}}
\newcommand{\eval}[2]{\left . #1 \right|_{#2}}
\newcommand{\seqnum}[1]{\href{https://oeis.org/#1}{\rm \underline{#1}}}
\setlength{\parskip}{0.5em}
\setlength{\parindent}{0pt}

\title{An Efficient Deterministic Primality Test: Proof}
\author{Joseph M. Shunia}
\date{May 2024}

\begin{document}
\maketitle

\section{Supporting Lemmas} \label{section:supportinglemmas}
We begin by presenting the supporting lemmas for our main theorem (\cref{section:maintheorem}).

\begin{lemma} \label{proof:carmichaelnumbersfail}
Let $n \in \Z_{>1}$ be a Carmichael number. Hence, $n = p_1 p_2 \cdots p_m$ is odd, composite, and squarefree, where the $p_i$ are distinct odd prime factors. Furthermore, $(p_i-1) \mid (n-1)$ for all $p_i \mid n$. Let $r \in \mathbb{P}_{\geq 3}$ be the least odd prime such that $r \nmid n (n-1)$. Let $(x^r-2, n)$ be the ideal generated by $x^r-2$ and $n$ in the polynomial ring $\Z[x]$. Consider the polynomial $f(x) := (x+1)^n - x^n - 1 \in \Z[x]$.

Suppose $x^n \not\equiv x \pmod{(x^r-2,n)}$. Then, $f(x) \not\equiv 0 \pmod{(x^r-2, n)}$.
\end{lemma}
\begin{proof}
The assumption $r \nmid n (n-1)$ implies that $r \nmid n$ and $r \nmid (n-1)$. We will begin by briefly show why this is necessary. Suppose $r \mid n$, therefore $r = p$ where $p \mid n$. Then
\begin{align*}
    x^n \equiv 2 \pmod{(x^r-2, p)}
    \quad \Longrightarrow \quad
    (x+1)^n \equiv 3 \pmod{(x^r-2, p)} .
\end{align*}
Hence, we have trivially
\begin{align*}
    f(x) \equiv (x+1)^n - x^n - 1 \equiv 0 \pmod{(x^r-2, p)} .
\end{align*}
Next, suppose $r \mid (n-1)$. Then $r \mid (p-1)$ for some $p \mid n$. Since $p$ is prime, $(p-1) = \phi(p)$. Leading to
\begin{align*}
    x^n \equiv x \pmod{(x^r-2, p)}
    \quad \Longrightarrow \quad
    (x+1)^n \equiv x+1 \pmod{(x^r-2, p)} .
\end{align*}
Again, it is easy to see that $f(x) \equiv 0 \pmod{(x^r-2, p)}$ in such case.

We will now show that $f(x) \equiv 0 \pmod{(x^r-2, n)}$ leads to a contradiction under the given conditions. Assume, for the sake of contradiction, that
\begin{align*}
    f(x) \equiv (x+1)^n - x^n - 1 \equiv 0 \pmod{(x^r-2, n)} .
\end{align*}

Since the congruence holds mod $(x^r-2, n)$, it must also hold mod $(x^r-2, p)$ for each prime factor $p$ of $n$. Otherwise, $n$ could not divide $f(x)$. Thus, for all primes $p \mid n$, we have
\begin{align*}
f(x) \equiv (x+1)^n - x^n - 1 \equiv (x+1)^p - x^p - 1 \equiv 0 \pmod{(x^r-2, p)} \\
\Longleftrightarrow (x+1)^n - x^n \equiv (x+1)^p - x^p \equiv 1 \pmod{(x^r-2, p)} .
\end{align*}

From this, we deduce
\begin{align*}
\left( (x+1)^{n/p} - x^{n/p} \right)^p \equiv 1 \pmod{(x^r-2, p)} .
\end{align*}

Leading to
\begin{align*}
\left( (x+1)^{n/p} - x^{n/p} \right)^p \equiv (x+1)^n - x^n \equiv (x+1)^p - x^p \equiv 1 \pmod{(x^r-2, p)} .
\end{align*}

This also implies
\begin{align*}
    \zeta_p \equiv (x+1)^{n/p} - x^{n/p} \pmod{(x^r-2, p)} ,
\end{align*}
where $\zeta_p$ is a $p$-th root of unity modulo $(x^r-2, p)$. By the Chinese Remainder Theorem (CRT), since the congruences hold mod $(x^r-2, p)$ for each prime factor $p$ of $n$, they also hold mod $(x^r-2, n)$. Thus, we have
\begin{align*}
\zeta_n &\equiv (x+1)^{n/n} - x^{n/n} \pmod{(x^r-2, n)} \\
&\equiv (x+1)^1 - x^1 \pmod{(x^r-2, n)} \\
&\equiv (x+1) - x \pmod{(x^r-2, n)} \\
&\equiv 1 \pmod{(x^r-2, n)} .
\end{align*}
This is consistent with $\zeta_p$ being a trivial $p$-th root of unity modulo $(x^r-2, n)$. That is
\begin{align*}
\zeta_p \equiv 1 \pmod{(x^r-2, p)} .
\end{align*}
Hence, we have
\begin{align*}
\left( (x+1)^{n/p} - x^{n/p} \right)^p \equiv (x+1)^{n/p} - x^{n/p} \equiv (x+1)^n - x^n \equiv (x+1)^p - x^p \equiv 1 \pmod{(x^r-2, p)} .
\end{align*}
Then, for each $p$, we must consider the following mutually exclusive cases:
\begin{enumerate}
\item[(i)] $x^p \equiv x^{n/p} \pmod{(x^r-2, p)} \quad\Longleftrightarrow\quad (x+1)^p \equiv (x+1)^{n/p} \pmod{(x^r-2, p)}$,
\item[(ii)] $x^n \equiv x^p \pmod{(x^r-2, p)} \quad\Longleftrightarrow\quad (x+1)^n \equiv (x+1)^p \pmod{(x^r-2, p)}$.
\end{enumerate}
Each case, taken individually, allows for $f(x) \equiv 0 \pmod{(x^r-2,p)}$. These cases are mutually exclusive, since satisfying both (i) and (ii) leads to
\begin{align*}
    x^{n/p} \equiv x^p \equiv x^n \pmod{(x^r-2, p)} ,
\end{align*}
implying that $p=r$ and $r \mid n$, contradicting the theorem.

Now, suppose cases (i) and (ii) are both false. If $x^n \equiv \zeta_p x \pmod{(x^r-2, p)}$, where $\zeta_p$ is a non-trivial $p$-th root of unity modulo $(x^r-2, p)$, then $(x+1)^n \equiv \zeta_p (x+1) \pmod{(x^r-2, p)}$. This is possible because the polynomial ring $\Z[x]/(x^r-2, p)$ is isomorphic to the direct product of fields $\mathbb{F}_p[x]/(x-\alpha_1) \times \cdots \times \mathbb{F}_p[x]/(x-\alpha_r)$, where the $\alpha_i$ are the roots of $x^r-2$ in an algebraic closure of $\mathbb{F}_p$. In some of these fields, there may exist non-trivial $p$-th roots of unity, allowing for this. However, we showed above that $\zeta_n \equiv 1 \pmod{(x^r-2, n)}$, so this would imply $x^n \equiv \zeta_n x \equiv x \pmod{(x^r-2, n)}$, contradicting the assumption in the theorem that $x^n \not\equiv x \pmod{(x^r-2, n)}$.

Finally, suppose either case (i) or (ii) is true for all primes $p \mid n$. For $n$, the two cases (i) and (ii) collapse to a single case, since $p$ is replaced by $n$ in the exponents when lifting via the CRT:
\begin{align*}
x^n \equiv x \pmod{(x^r-2, n)} \quad\Longleftrightarrow\quad (x+1)^n \equiv x+1 \pmod{(x^r-2, n)} .
\end{align*}
However, this is a contradiction, since again, $x^n \not\equiv x \pmod{(x^r-2, n)}$ by assumption in the theorem. Therefore $f(x) \not\equiv 0 \pmod{(x^r-2, n)}$. This completes the proof.
\end{proof}

\begin{lemma} \label{proof:niscarmichaelnumber}
Let $n, r \in \Z_{>1}$ such that $n$ is odd, $r \geq 3$, and $r \nmid n$. Consider the polynomial
\begin{align*}
f(x) := (x+1)^n - x^n - 1 \in \Z[x] .
\end{align*}
Let $(x^r-2, n)$ be the ideal generated by $x^r-2$ and $n$ in the polynomial ring $\Z[x]$. Suppose
\begin{align*}
f(x) \equiv 0 \pmod{(x^r-2, n)} .
\end{align*}
Then it is necessary, but not sufficient, that $n$ is a Carmichael number.
\end{lemma}
\begin{proof}
Let $p \mid n$ be prime. Consider
\begin{align*}
    f(x) \equiv (x+1)^n - x^n - 1 \equiv 0 \pmod{(x^r-2, p)} .
\end{align*}
Expanding the term $(x+1)^n$ via the Binomial Theorem and simplifying, we see
\begin{align*}
    f(x) = \sum_{k=1}^{n-1} \binom{n}{k} x^k .
\end{align*}
From Lucas Theorem, since $p \not= n$, we know that $p$ cannot divide all $\binom{n}{k}$. Furthermore, since $r \geq 3$, the reduction of $f(x)$ mod $(x^r-2)$ will leave a polynomial remainder of degree $d \geq 1$. In other words, $f(x)$ is not a constant modulo $(x^r-2)$. Then, for $f(x)$ to be zero, it must vanish for all $a \in \Z$ when evaluated modulo $p$. That is, 
\begin{align*}
    \forall a \in \Z, \quad f(a) \equiv (a+1)^n - a^n - 1 \equiv 0 \pmod{p} .
\end{align*}
Re-arranging this, we have
\begin{align*}
    \forall a \in \Z, \quad f(a) \equiv (a+1)^n \equiv a^n + 1 \pmod{p} .
\end{align*}
Thus,
\begin{align*}
    \forall a \in \Z, \quad a^n \equiv a \pmod{p} .
\end{align*}
By Fermat's Little Theorem (FLT), this implies $(p-1) \mid (n-1)$. Since this must be true for all primes $p \mid n$, we conclude that $n$ must be a Carmichael number. However, this condition is not sufficient, as we did not prove the conditions under which $f(x) \equiv 0 \pmod{(x^r-2, n)}$ holds for Carmichael numbers.
\end{proof}

\begin{lemma} \label{proof:carmichalenumberxnnotequalsx}
Let $n \in \Z_{>1}$ be a Carmichael number. Hence, $n$ is odd, composite, and squarefree. Let $r \in \mathbb{P}_{\geq 3}$ be the least odd prime such that $r \nmid n (n-1)$. Let $(x^r-2, n)$ be the ideal generated by $x^r-2$ and $n$ in the polynomial ring $\Z[x]$. Then, $x^n \not\equiv x \pmod{(x^r-2,n)}$.
\end{lemma}
\begin{proof}
The reduction of $x^n \pmod{(x^r-2)}$ leaves a polynomial remainder of degree $d = (n \bmod r)$. Formally, we have
\begin{align*}
    x^n \equiv 2^{\floor{n/r}} x^d \equiv 2^{\floor{n/r}} x^{n \bmod r} \pmod{(x^r-2)}
\end{align*}
Since $n = p_1 p_2 \cdots p_m$ is a Carmichael number, we know that $(p_i-1) \mid (n-1)$ for all prime factors $p_i \mid n$. In such case
\begin{align*}
    r \nmid (n-1) \quad \Longrightarrow \quad n \not\equiv 1 \pmod{r} . 
\end{align*}
Thus, $d \not= 1$. Leading to
\begin{align*}
    x^n \not\equiv x \pmod{(x^r-2, n)} .
\end{align*}
\end{proof}

\section{Main Theorem} \label{section:maintheorem}
\begin{theorem} \label{proof:main}
Let $n \in \Z_{>1}$ such that $n$ is odd. Let $r \in \mathbb{P}_{\geq 3}$ be the least odd prime such that $r \nmid n (n-1)$. Let $(x^r-2, n)$ be the ideal generated by $x^r-2$ and $n$ in the polynomial ring $\Z[x]$. Consider the polynomial $f(x) := (x+1)^n - x^n - 1 \in \Z[x]$. Suppose $f(x) \equiv 0 \pmod{(x^r-2, n)}$. Then, $n$ is prime.
\end{theorem}
\begin{proof}
Expanding $f(x)$ by the Binomial Theorem and simplifying, we see
\begin{align*}
    f(x) = (x+1)^n - x^n - 1 = \sum_{k=1}^{n-1} \binom{n}{k} x^k .
\end{align*}
If $n$ is prime, then $\binom{n}{k} \equiv 0 \pmod{n}$ for all $k$ in the sum. Clearly then, $f(x) \equiv 0 \pmod{n}$ when $n$ is prime.

On the other hand, suppose $n$ is composite. In this case, Lucas Theorem tells us that $n$ cannot possibly divide all $\binom{n}{k}$ in the sum. Applying \cref{proof:niscarmichaelnumber}, we see that $n$ must be a Carmichael number for $f(x)$ to be zero. Furthermore,  by \cref{proof:carmichalenumberxnnotequalsx}, since $r \geq 3$ and $r \nmid (n-1)$, we have $x^n \not\equiv x \pmod{(x^r-2,n)}$. Under these constraints, we have
\begin{align*}
    f(x) \equiv (x+1)^n - x^n - 1 \not\equiv 0 \pmod{(x^r-2, n)} \quad \text{(By \cref{proof:carmichaelnumbersfail})} .
\end{align*}
Therefore, under the given conditions, if $f(x) \equiv 0 \pmod{(x^r-2, n)}$, then $n$ is prime.
\end{proof}

\end{document}