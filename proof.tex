\documentclass{article}
\usepackage{amsmath}
\usepackage{amsthm}
\usepackage[utf8]{inputenc}
\title{Primality Test Proof}
\author{Joseph M. Shunia}
\date{July 20, 2023}
\begin{document}
\maketitle
\newtheorem{theorem}{Theorem}
\begin{theorem}
Let $n$ be an odd integer $>3$.
Let $D$ be the least integer $>2$ which does not divide $n-1$.
If $2^{\left\lfloor (n-1)/D \right\rfloor} + 1 \equiv (2^{\left\lfloor (n-1)/D \right\rfloor} + 1)^{n} \equiv \sum_{k=0}^{n} \binom{n}{k}2^{\left\lfloor (k-1)/D \right\rfloor} \pmod{n}$, then $n$ is prime.
\end{theorem}
\begin{proof}
Define the function $a(k) = 2^{\left\lfloor (k-1)/D \right\rfloor}$, then $2^{\left\lfloor (n-1)/D \right\rfloor} + 1 \equiv (2^{\left\lfloor (n-1)/D \right\rfloor} + 1)^{n} \pmod{n}$ can be rewritten as $a(n) + 1 \equiv (a(n) + 1)^{n} \pmod{n}$. This means that either $n$ is prime or $n$ is a Fermat psuedoprime to base $a(n) + 1$. However, we also have that $(a(n) + 1)^{n} \equiv \sum_{k=0}^{n} \binom{n}{k}2^{\left\lfloor k/D \right\rfloor} \pmod{n}$
which is equivalent to $(a(n) + 1)^{n} \equiv \sum_{k=0}^{n} \binom{n}{k}a(k) \pmod{n}$.

By expanding $(a(n) + 1)^{n}$ using the binomial theorem, we get $\sum_{k=0}^{n} \binom{n}{k}a(n)^k$, and it is easy to see that this is equal to the right hand side $\sum_{k=0}^{n} \binom{n}{k}a(k)$ if and only if $a(n)^k \equiv a(k) \pmod{n}$ for all $0 \leq k \leq n$.

Note that $a(n)^k = 2^{k \left\lfloor \frac{n-1}{D} \right\rfloor}$. On the other hand, $a(k) = 2^{\left\lfloor \frac{k-1}{D} \right\rfloor}$, so $a(n)^k \equiv a(k) \pmod{n}$ if and only if $2^{k \left\lfloor \frac{n-1}{D} \right\rfloor} \equiv 2^{\left\lfloor \frac{k-1}{D} \right\rfloor} \pmod{n}$ for all $0 \leq k \leq n$. This is only possible if $D$ is a divisor of $n-1$ or $n$ is prime. Since we assumed that $D$ does not divide $n-1$, the only possibility is that $n$ is prime. Therefore, the theorem is proved.
\end{proof}
\end{document}