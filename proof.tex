\begin{proof}
We aim to prove that for an odd integer $n > 3$ such that $2^{n-1} \equiv 1 \pmod{n}$, and $D$ being the least integer greater than 1 which does not divide $n-1$, if the congruence $2^{\lfloor\frac{n-1}{D}\rfloor} + 1 = \sum_{k=0}^{n}\binom{n}{k}2^{\lfloor\frac{k}{D}\rfloor} \pmod{n}$ holds, then $n$ is prime or GCD($a(n) - 1$, $n$) is a non-trivial factor of $n$, where $a(n) = 2^{\lfloor\frac{n-1}{D}\rfloor}$.

\textbf{Step 1: Proof that $2^{\lfloor\frac{n-1}{D}\rfloor} \not\equiv 1 \pmod{n}$:}

Given $2^{n-1} \equiv 1 \pmod{n}$, by the properties of the order of an integer modulo $n$, the smallest $k$ such that $2^k \equiv 1 \pmod{n}$ must be $k = n-1$ or a divisor of $n-1$.

Since $\lfloor\frac{n-1}{D}\rfloor$ is strictly less than $n-1$ and $D$ does not divide $n-1$, it follows that $2^{\lfloor\frac{n-1}{D}\rfloor}$ can't be equivalent to $1 \pmod{n}$.

\textbf{Step 2: Proof for GCD statement:}

By the binomial theorem and substituting with $a(n)$ in:
\[
2^{\lfloor\frac{n-1}{D}\rfloor} + 1 = \sum_{k=0}^{n}\binom{n}{k}2^{\lfloor\frac{k}{D}\rfloor} \pmod{n}
\]
We see that:
\[
a(n) + 1 \equiv (1 + a(n))^n \mod n
\]
Now, by Fermat's little theorem, we know that if $n$ is prime and $\gcd(a(n), n) = 1$, then $(a(n))^n \equiv a(n) \pmod{n}$.

Therefore, we can rewrite the above congruence as:
\[
(a(n))^n + 1 \equiv (a(n))^2 + a(n) + 1 \pmod{n}
\]
This implies that $(a(n))^n - (a(n))^2 - a(n)$ is divisible by $n$. 

If we factor this expression, we get:
\[
(a(n) - 1)(a(n)^{n-1} + a(n)^{n-2} + ... + a(n) + 1)
\]
So, either $(a(n) - 1)$ or $(a(n)^{n-1} + a(n)^{n-2} + ... + a(n) + 1)$ is divisible by $n$. 

If $(a(n) - 1)$ is divisible by $n$, then $\gcd(a(n) - 1, n)$ is a non-trivial factor of $n$. 

If $(a(n)^{n-1} + a(n)^{n-2} + ... + a(n) + 1)$ is divisible by $n$, then we can use the fact that $(a(n))^2 + a(n) + 1$ is also divisible by $n$ to show that $\gcd(a(n)^2 - a(n), n)$ is also a non-trivial factor of $n$. 

To see this, note that:
\[
(a(n)^{n-3} - a(n))(a(n)^2 + a(n) + 1) = (a(n)^{n-3})(a(n)^2 - a(n)) - (a(n))^3
\]
Since both terms on the right-hand side are divisible by $n$, so is their difference. This means that $(a(n)^{n-3} - a(n))$ is divisible by $n$. 

Repeating this process, we can show that $(a(n)^{k} - a(k))$ is divisible by $n$ for any positive integer $k < n$. In particular, for $k = 2$, we get that $(a(2) - a(2)) = (a(2))^2 - a(2)$ is divisible by $n$. 

Hence, $\gcd(a(n)^2 - a(n), n)$ is a non-trivial factor of $n$. 

Therefore, in either case, we have shown that if the congruence holds, then $n$ is prime or $\gcd(a(n) - 1, n)$ is a non-trivial factor of $n$.
\end{proof}