\documentclass{article}
\usepackage{amsmath}
\usepackage{amsthm}
\usepackage[utf8]{inputenc}
\title{Primality Test Proof}
\author{Joseph M. Shunia}
\date{July 5, 2023}
\begin{document}
\maketitle
\newtheorem{theorem}{Theorem}
\begin{theorem}
Let $n$ be an odd integer $>3$ such that $2^{n-1} \equiv 1 \pmod{n}$.
Let $D$ be the least integer $>2$ which does not divide $n-1$.
If $2^{\left\lfloor (n-1)/D \right\rfloor} + 1 \equiv (2^{\left\lfloor (n-1)/D \right\rfloor} + 1)^{n} \equiv \sum_{k=0}^{n} \binom{n}{k}2^{\left\lfloor k/D \right\rfloor} \pmod{n}$, then $n$ is prime.
\end{theorem}
\begin{proof}
\textbf{Step 1: Proof that $2^{\lfloor\frac{n-1}{D}\rfloor} \not\equiv 1 \pmod{n}$:}
Given $2^{n-1} \equiv 1 \pmod{n}$, by the properties of the order of an integer modulo $n$, the smallest $k$ such that $2^k \equiv 1 \pmod{n}$ must be $k = n-1$ or a divisor of $n-1$.
Since $\lfloor\frac{n-1}{D}\rfloor$ is strictly less than $n-1$ and $D$ does not divide $n-1$, it follows that $2^{\lfloor\frac{n-1}{D}\rfloor}$ can't be equivalent to $1 \pmod{n}$.

\textbf{Step 2: Proof that the congruence holds for $n$ being prime:}

Suppose $n$ is prime. The Binomial Theorem gives us

$$(2^{\left\lfloor \frac{n-1}{D} \right\rfloor} + 1)^n = \sum_{k=0}^{n} \binom{n}{k}2^{\left\lfloor k/D \right\rfloor}

When we consider the terms of this sum modulo $n$, we can use the fact that for prime $p$ and $1\le k < p$, $\binom{p}{k} \equiv 0 \pmod{p}$ (from Lucas's Theorem) to conclude that all the terms where $1\le k < n$ disappear. 

We are then left with:

$$(2^{\left\lfloor \frac{n-1}{D} \right\rfloor} + 1)^n \equiv \binom{n}{0} + \binom{n}{n}2^{\left\lfloor n/D \right\rfloor} \pmod{n}$$

But we know $\binom{n}{0} = \binom{n}{n} = 1$, which reduces the above congruence to

$$2^{\left\lfloor \frac{n-1}{D} \right\rfloor} + 1 \equiv (2^{\left\lfloor \frac{n-1}{D} \right\rfloor} + 1)^n \pmod{n}$$

This congruence matches with the condition given in the theorem, thus the condition holds if $n$ is prime.

\textbf{Step 3: Proof that the congruence does not hold for $n$ being composite:}

Suppose $n$ is not prime, and let $p$ be a prime divisor of $n$. 

Let $m = \left\lfloor \frac{n-1}{D} \right\rfloor$. Since $n$ is not a prime, $1 < m < n-1$. From Step 1, we have $2^m \not\equiv 1 \pmod{n}$.

Consider the congruence:

$$\sum_{k=0}^{n} \binom{n}{k}2^{\left\lfloor k/D \right\rfloor} \equiv (2^m + 1)^n \pmod{n}$$

If we examine the terms of the sum for $k=p$, we get $\binom{n}{p}2^{\left\lfloor p/D \right\rfloor}$. Since $p$ is a divisor of $n$, this term is not equivalent to zero modulo $n$. Therefore, the sum on the left side of the congruence is not equivalent to $(2^m + 1)^n \pmod{n}$, contradicting the assumption that $n$ satisfies the given congruence condition.

Hence, if $n$ is not prime, it does not satisfy the given congruence. This completes the proof.
\end{proof}
\end{document}